\chapter{Risk report}\label{chap:risk}

In order to perform risk identification and analysis, we are adopting Sommerville's checklist\footnote{Ian~Sommerville. \emph{Software Engineering}. 10th edition. Pearson, 2015.}, as we believe it to be a good balance of compactness, comprehensiveness and intuitiveness. 

There are six types of risk that may be considered, and they are analysed in the following sections.





\section{Technology risks}

These risks derive from the software and hardware technologies that are used to develop the system. 

We can identify a number of them:
\begin{itemize}
	\item The database cannot process as many transactions per second as expected. To overcome this problem, the possibility of buying a higher-performance database should be considered.
	\item Reusable software components need severe refactoring. In-depth analysis should be performed before using already written code, and only valid components should be adopted.
	\item Data corruption in the communication between clients and servers occurs more frequently than expected. Protocols to improve the reliability of the communications should be introduced.
	\item The system may suffer from outages, due to power outages, hardware failures. System redundancy may be of help, in order to contrast this kind of risk.

\end{itemize}








\section{People risks}

People risks are associated with the people in the development team.

Trivially, we have the following:
\begin{itemize}
	\item Key staff are ill and unavailable at critical times: since our team is currently composed of two people, any absence would almost certainly cause delays, unless our work is reorganised so that there is more overlap, and thus we get more insight into each other's jobs.
\end{itemize}





\section{Organisational risks}

Risks deriving from the organisational environment where the software is being developed are grouped in this category.

The following is the main risks in this category:
\begin{itemize}
	\item Financial difficulties due to a reduced income from the municipality; however the city council's commitment is strong, so reductions in the project budget are very unlikely.
\end{itemize}






\section{Tools risks}

This kind of risks derive from the software tools and other support software used to develop the system.

In the previous documents, limited specification was made about the tools to be used. No relevant risk relating to this category can be pointed out.




\section{Requirements risks}

Risks that derive from changes to the customer requirements fall into this category.

The two main requirements risks we identify are:
\begin{itemize}
	\item Changes to requirements that require major design rework may be proposed. This is totally beyond our control. The only countermeasure is to deliver a detailed requirement analysis document and thoroughly discuss it with the municipality. Also, deriving traceability information to assess requirements changes may be of help.
	\item Legislation concerning taxi service and privacy may change in unexpected ways; hardly we can do something to predict the probability and the impact on our project.
\end{itemize}


	




\section{Estimation risks}

Here we collect risks that derive from the management estimates of the resources required to build the system.

Only one risk can be categorised here:
\begin{itemize}
	\item This risk depends mainly on the accuracy of the estimations in \cref{chap:estimation}, and can lead to an insufficient money or time allocation for the development of the project. Thorough investigations should be performed to check the correctness of the values.
\end{itemize}








	
