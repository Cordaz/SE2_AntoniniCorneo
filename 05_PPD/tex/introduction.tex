\chapter{Introduction}\label{chap:introduction}


\section{Purpose and scope}
After the long planning phase which myTaxiService underwent in the past months, it seems appropriate to define a precise planning for its development, as well as identify its supposed cost and the needed effort to implement it.

With this document we aim to reach this goal.



\section{References}\label{sec:references}
In the following chapters we will refer to the following documents:\begin{itemize}
	
	\item \emph{COCOMO II.2000 model manual}: \url{http://csse.usc.edu/csse/research/COCOMOII/cocomo2000.0/CII_modelman2000.0.pdf}.
	
	\item \emph{Function Point Languages Table}: \url{http://www.qsm.com/resources/function-point-languages-table}.

\end{itemize}

Moreover, it is advisable to consider also the previous documents relating to myTaxiService:\begin{itemize}
	
	\item \emph{Assignments 1 and 2}, section~2, parts~I and~II; \printdate{2015-10-13}. This is to be regarded as the high level project description. Available at: \url{https://beep.metid.polimi.it/documents/3343933/d5865f65-6d37-484e-b0fa-04fcfe42216d}.
	
	\item \emph{Requirement analysis and specification document}; \printdate{2015-11-6}. Available at: \url{https://github.com/Cordaz/SE2_AntoniniCorneo/raw/master/Deliveries/1_RASD.pdf}.
	
	\item \emph{Design document}; \printdate{2015-12-4}. Available at: \url{https://github.com/Cordaz/SE2_AntoniniCorneo/raw/master/Deliveries/2_DD.pdf}.
	
	\item \emph{Integration test plan document}; \printdate{2016-1-21}. Available at: \url{https://github.com/Cordaz/SE2_AntoniniCorneo/raw/master/Deliveries/4_ITPD  .pdf}.
	 
\end{itemize}



\section{Overview of the document} 
This document develops as follows.

In \cref{chap:estimation} Function Points and COCOMO methods are applied in order to estimate the effort and cost to develop the project; the justification for some choices we made is left to \cref{chap:justification}, in order no to clutter \cref{chap:estimation} itself. \Cref{chap:allocation} deals with the scheduling of tasks and allocation of resources. Finally, in \cref{chap:risk} we report the main risks the development of the project may encounter. 





