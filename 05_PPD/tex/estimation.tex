\chapter{Cost and effort estimation}\label{chap:estimation}

%%%%%%%%%%%%%%%%%%%%%%%%%%%%%%%%%%%%%%%%%%%%%%%%%%%%%%%%%%%%%%%%%%%%%%%%%%%%%%
% TODO	Ripensa riga TOTAL!
%%%%%%%%%%%%%%%%%%%%%%%%%%%%%%%%%%%%%%%%%%%%%%%%%%%%%%%%%%%%%%%%%%%%%%%%%%%%%%


In this chapter we are going to estimate the size of myTaxiService project by means of the Function points method, which is a language-independent way of quantifying program functionality.

Then, after converting Function points into number of lines of source code, we are able to derive a development effort estimation based on system requirements and design specification, in terms of time and number of people. This is possible thanks to the formulas elaborated in \mbox{COCOMO II.2000} (\emph{Constructive Cost Model}), which we are going to present in the next sections.


\newcommand{\mSize}{\mathbf{s}}
\newcommand{\mFP}{\text{FP}}
\newcommand{\mEffort}{\mathbf{e}}
\newcommand{\eaf}{\text{EAF}}  % Effort Adjustment Factor
\newcommand{\myW}{1.5cm}


\section{Function points}

The Function points approach tries to measure the program functionalities, by quantifying the information processing functionality associated with major external data or control input, output, or file types. Five user function types are identified. At the end of estimation, the total number of function points ($ \mFP $) is given by the following formula:
\begin{equation}
\mFP = \sum \text{\#functions\_by\_type} * \text{weight} \label{eqn:fp}
\end{equation}


In \cref{tab:FPweights} the weights for each function type, according to their complexity, are presented, as a reference. Then, in the following paragraph, each type is analysed and the corresponding function points are calculated.


\begin{table}\begin{tabularx}{\textwidth}{ >{\ttfamily}C{.6cm} X C{\myW} C{\myW} C{\myW} }

\toprule
	\multirow{2}{*}{\normalfont\textsc{ID}} &
	\multirow{2}{*}{\normalfont\textsc{Function type}} & \multicolumn{3}{c}{\normalfont\textsc{Complexity weight}}\\
	\cmidrule{3-5}%
	&& \itshape{Simple} & \itshape{Average} & \itshape{Complex} \\
\toprule
	EI & External input				& $3$ 	& $4$	& $6$ \\
\midrule
	EO & External output				& $4$	& $5$	& $7$ \\
\midrule
	ILF & Internal logical file		& $7$	& $10$	& $15$ \\
\midrule
	EIF & External interface files	& $5$	& $7$	& $10$ \\
\midrule
	EQ & External inquiry			& $3$ 	& $4$	& $6$ \\
\bottomrule

\end{tabularx}

\caption{Function points complexity weights table.}
\label{tab:FPweights}

\end{table}










\subsection*{External input}

External input functions deal with unique user data or user control input type that enters the external boundary of the software system being measured.

Users interact with myTaxiService system in many ways:
\begin{itemize}
	
	\item Login, logout and registration: these are simple operations managed by simple components.
	
	\item Memorise an address: it is a simple operation.
	
	\item Make a request, reservation: for each request or reservation the systems must interact with many entities. Moreover a taxi has to be allocated. These are complex operations, since many entities are involved.
	
	\item Cancel a reservation: a simple operation that requires only an object elimination. 
	
	\item Report taxi availability: this operation requires looking for the pertinence area of the taxi, in order to correctly enqueue it; this can be estimated as an average operation.
	
	\item Taxi driver accepts or refuses a ride: the first case involves two simple operations, since the ride shall be initiated and a notification must be prepared. Instead, the second case is more complicated: whenever a ride is refused, the system makes a new allocation.

\end{itemize}

\newcommand{\myWFP}{2.5cm}

\begin{table*}\begin{tabularx}{\textwidth}{ X >{\itshape}C{\myWFP} C{\myWFP} C{\myWFP} }

\toprule
	
	\normalfont\textsc{Function} &
	\normalfont\textsc{Complexity}	& 
	\normalfont\textsc{Number} &
	\normalfont\textsc{Total weight} \\

\toprule

	Login, logout, registration	& Simple		& $ 3 $		& $ 3*3 = 9 $ \\
\midrule
	Address memorisation			& Simple		& $ 1 $		& $ 1*3 = 3 $ \\
\midrule
	Request, reservation			& Complex	& $ 3 $		& $ 3*6 = 18 $ \\
\midrule
	Cancel reservation			& Simple		& $ 1 $ 		& $ 1*3 = 3 $ \\
\midrule
	Report availability			& Average	& $ 1 $ 		& $ 1*4 = 4 $ \\
\midrule
	Ride accepted				& Simple		& $ 2 $ 		& $ 2*3 = 6 $ \\
\midrule
	Ride refused				& Complex	& $ 1 $ 		& $ 1*6 = 6 $ \\

\bottomrule

\normalfont\textsc{Total} \texttt{EI} && & $ 49 $ \\

\bottomrule

\caption{External input summary table.}

\label{tab:ei}\end{tabularx}\end{table*}










\subsection*{External output}

This kind of functions counts each unique user data or control output type that leaves the external boundary of the software system.

myTaxiService system shall provide notification to all types of users, which can be of various type (SMS, push, \dots). Each notification contains a small amount of data, however those must comply with protocols, which makes the operation a bit more complex: an average cost is to be considered.



\begin{table*}\begin{tabularx}{\textwidth}{ X >{\itshape}C{\myWFP} C{\myWFP} C{\myWFP} }

\toprule
	
	\normalfont\textsc{Function} &
	\normalfont\textsc{Complexity}	& 
	\normalfont\textsc{Number} &
	\normalfont\textsc{Total weight} \\

\toprule

	Notification	& Average		& $ 1 $		& $ 1*5 = 5 $ \\

\bottomrule

\normalfont\textsc{Total} \texttt{EO} && & $ 5 $ \\
\bottomrule

\caption{External output summary table.}

\label{tab:eo}\end{tabularx}\end{table*}



\subsection*{Internal logical file}

Internal logical files are the major logical group of user data or control information in the software system as a logical internal file type. These include each logical file that is generated, used, or maintained by the software system.

Our system includes many ILFs, as they are needed to store all information about customers, areas and addresses, cabs and drivers, requests, reservations and allocations. In detail:

\begin{itemize}
	\item few information about customers are stored, in simple structures: first and last name, login credential and phone number. However, since registered customers' addresses are stored as well, it is appropriate to consider an average cost.
	\item to map the whole city in addresses and in their pertinence areas few data are needed: string and GPS data types may be sufficient to the purpose. However, every area contains a great amount of addresses, so an high complexity is to be accounted for.
	\item although many information about taxi drivers and taxi cabs are stored in the system, data needed for the operation of the system are more limited and very simple. Given that, we assign a low complexity level.
	\item the system needs to store many information for each request, reservation and many other information are needed to allocate the taxi. This requires a high complexity operation.
\end{itemize}






\begin{table*}\begin{tabularx}{\textwidth}{ X >{\itshape}C{\myWFP} C{\myWFP} C{\myWFP} }

\toprule
	
	\normalfont\textsc{Function} &
	\normalfont\textsc{Complexity}	& 
	\normalfont\textsc{Number} &
	\normalfont\textsc{Total weight} \\

\toprule

	Customers	& Average	& $ 1 $		& $ 1*10 = 10 $ \\
\midrule
	Areas and addresses		& Complex	& $ 2 $		& $ 2*15 = 30 $ \\
\midrule
	Taxis, taxi drivers		& Simple		& $ 1 $		& $ 1*7 = 7 $ \\
\midrule
	Requests, reservations, allocations	& Complex	& $ 3 $ 	& $ 3*15 = 45 $ \\

\bottomrule

\normalfont\textsc{Total} \texttt{ILF} && & $ 92 $ \\

\bottomrule

\caption{Internal logical files summary table.}

\label{tab:ilf}\end{tabularx}\end{table*}





\subsection*{External interface files} 

Files passed or shared between software systems should be counted as external interface file types within each system.

As a matter of fact, myTaxiService system does not require any external data, so this is set to $ 0 $.







\subsection*{External inquiry}
 
Here we count each unique input-output combination, where input causes and generates an immediate output, as an external inquiry type.

Registered customers can access their profile and their memorised addresses. Both operations can be considered as simple.


\begin{table*}\begin{tabularx}{\textwidth}{ X >{\itshape}C{\myWFP} C{\myWFP} C{\myWFP} }

\toprule
	
	\normalfont\textsc{Function} &
	\normalfont\textsc{Complexity}	& 
	\normalfont\textsc{Number} &
	\normalfont\textsc{Total weight} \\

\toprule

	Profile				& Simple		& $ 1 $		& $ 1*3 = 3 $ \\
\midrule
	Memorise address	& Simple		& $ 1 $		& $ 1*3 = 3 $ \\


\bottomrule

\normalfont\textsc{Total} \texttt{EQ} && & $ 6 $ \\



\bottomrule

\caption{External inquiry summary table.}

\label{tab:eq}\end{tabularx}\end{table*}





\subsection{Total function points}

The results from previous sections (\crefrange{tab:ei}{tab:eq}) are summarised in \cref{tab:FPtotal}.

\begin{table}\begin{tabularx}{\textwidth}{ >{\ttfamily}C{.6cm} X C{\myW} }

\toprule
\normalfont\textsc{ID} & \normalfont\textsc{Function type} & $ \mFP $\\

\toprule

	EI & External input				& $49$ \\
\midrule
	EO & External output				& $5$ \\
\midrule
	ILF & Internal logical file		& $92$ \\
\midrule
	EIF & External interface files	& $0$ \\
\midrule
	EQ & External inquiry			& $6$\\
\bottomrule

\end{tabularx}

\caption{Total function points.}
\label{tab:FPtotal}

\end{table}



By applying \cref{eqn:fp}, we get the total number of Function points:
\begin{equation}
	\mFP = 152
\end{equation}

Those we obtained are the Unadjusted function points (UFP). An attempt to adapt this result (which focuses on functionality) to the prediction of development costs was made, but as it was wittily noted, ``it is like correcting the tallness of a person to relate it to a measure of his/her intelligence'' (N. Fenton), so we are not going down this road.






















\section{COCOMO}\label{sec:cocomo}

In the following we are going to adopt the formulas defined in \mbox{\emph{COCOMO II.2000}} standard. Only a brief introduction of the theoretic meaning of the formulas is given. For the sake of compactness, we are not going to present the whole tables from which values are extracted\footnote{Refer to the documents mentioned in \cref{sec:references} for an insight on COCOMO model.}.  


\subsection{Sizing the project}

A reasonable size estimate of a project is very important for a good model estimation. However, at this stage this is a challenging operation, since it is nearly impossible to correctly guess how much of myTaxiService code will be newly written, and how much, instead, will be reused or readapted.

That is why we are going to provide the (very unlikely) worst-case estimation: we assume that the whole code will be written from scratch. The quantity $\mSize$ is the estimated number of source lines of code (SLOC):
\begin{equation}
%
	\mSize = p * \mFP
%
\end{equation}

Parameter $ p = 46 $ converts the function points to SLOC, in the specific case of J2EE\footnote{See \cref{sec:references} for the source of this figure.} (Java Enterprise Edition).

In our case the sizing of the project results:
\begin{equation}
%
	\mSize = 46 * 152 = 6992
%
\end{equation}



\subsection{Effort estimation}

Effort is expressed in terms of \mbox{person-months} (PM). A \mbox{person-month} is the amount of time one person spends working on the software development project for one month. COCOMO defines the following formula to estimate it:
\begin{equation}
%
	\mEffort = 2.94 * { \left( \frac{\mSize}{1000} \right) }^{E} * \eaf \label{eqn:effort}
%
\end{equation}


Exponent $ E $ and parameter $ \eaf $ are defined as follows:
\begin{gather}
%
	%E = exponent derived from Scale Drivers
	E = 0.91 + 0.01 * \sum_{i}^{5} \text{SF}_i \label{eqn:Eexp}\\%
%
	% EAF = Effort Adjustment Factor derived from Cost Drivers
	\eaf     = \prod_{i=1}^{n} \text{EM}_i \label{eqn:eaf}
%
\end{gather}




In the following paragraphs we are going to present the factors from which exponent $ E $ and parameter $ \eaf $ are derived. For a justification of the choices we made, refer to \cref{chap:justification}.









\subsection*{Scale factors} Exponent $ E $ is the aggregation of five scale factors that account for some possible overheads encountered for software projects. Each factor has a range of rating levels, from \emph{Very low} to \emph{Extra high}, and each rating level has a weight. The specific value of the weight is called a scale factor ($ \text{SF} $). In \cref{tab:sf} we are presenting the values we adopted.



\begin{table}\begin{tabularx}{\textwidth}{ >{\ttfamily}c X >{\itshape}c c }

\toprule
\normalfont\textsc{ID} & \normalfont\textsc{Scale factor} & \normalfont\textsc{Level} & $ \text{SF} $ \\
\toprule
PREC	& Precedentedness			& Low		& $ 4.96 $ \\ \midrule
FLEX	& Development flexibility	& High		& $ 2.03 $ \\ \midrule
RESL	& Risk resolution			& Nominal	& $ 4.24 $\\ \midrule
TEAM	& Team cohesion				& Very high	& $ 1.10 $\\ \midrule
PMAT	& Process maturity			& High		& $ 3.12 $\\ 

\bottomrule
	
\end{tabularx}

\caption{Scale factors summary table.}
\label{tab:sf}

\end{table}



As a result of the previous choices and on the basis of \cref{eqn:Eexp}, the parameter $ E $ results:
\begin{equation}
	E = 0.91 + 0.01 * 15.45 = 1.0645 \label{eqn:expVAL}
\end{equation}


















\subsection*{Cost drivers} The parameter $ \eaf $ is derived from the effort multipliers ($ \text{EM} $) of the Cost drivers. Cost drivers are used to capture characteristics of the product under development, of the personnel working on it, and of general practices that affect the effort to complete the project. 

Again, in \cref{tab:cd} we present only the summary of our evaluation of the parameters.



\begin{table}\begin{tabularx}{\textwidth}{ >{\ttfamily}c X >{\itshape}c c }

\toprule
\normalfont\textsc{ID} & \normalfont\textsc{Cost factor} & \normalfont\textsc{Level} & $ \text{EM} $ \\
\toprule

RELY	& Required software reliability	& Low		& $ 0.92 $ \\ \midrule
DATA	& Data base size					& Nominal	& $ 1.00 $ \\ \midrule
CPLX	& Product complexity				& High		& $ 1.17 $ \\ \midrule
RUSE	& Developed for reusability		& High		& $ 1.07 $ \\ \midrule
DOCU	& Documentation for lifecycle needs		& High		& $ 1.11 $ \\ \midrule

TIME	& Execution time constraint		& --			&    --    \\ \midrule
STOR	& Main storage constraint		& --			&    --    \\ \midrule
PVOL	& Platform volatility			& Low		& $ 0.87 $ \\ \midrule

ACAP	& Analyst capability				& High		& $ 0.85 $ \\ \midrule
PCAP	& Programmer capability			& High		& $ 0.88 $ \\ \midrule
PCON	& Personnel continuity			& Very low	& $ 1.29 $ \\ \midrule
APEX	& Applications experience 		& Nominal	& $ 1.00 $ \\ \midrule
PLEX	& Platform Experience			& Nominal	& $ 1.00 $ \\ \midrule
LTEX	& Language and tool experience	& Nominal	& $ 1.00 $ \\ \midrule

TOOL	& Use of software tools			& Nominal	& $ 1.00 $ \\ \midrule
SITE	& Multisite development			& Extra high	& $ 0.80 $ \\ \midrule
SCED	& Required development schedule	& High		& $ 1.00 $ \\



\bottomrule
	
\end{tabularx}

\caption{Cost drivers summary table.}
\label{tab:cd}


\end{table}


%%%%%%%%%%%%%%%%%%%%%%%%%%%%%%%%%%%%%%%%%%%%%%%%%%%%%%%%%%%%%%%%%%%%%%%%%%%%%%
\clearpage
%%%%%%%%%%%%%%%%%%%%%%%%%%%%%%%%%%%%%%%%%%%%%%%%%%%%%%%%%%%%%%%%%%%%%%%%%%%%%%



As a result of the previous choices, parameter $ \eaf $, defined in \cref{eqn:eaf}, results:

\begin{equation}
	\eaf  = 0.8586	 \label{eqn:eafVAL}
\end{equation}





\subsection*{Effort estimation}

Thanks to the results in \cref{eqn:expVAL,eqn:eafVAL}, we are now able to estimate the value of the effort, defined in \cref{eqn:effort}:
\begin{equation}
%
	\mEffort = 2.94 * { \left( \frac{6992}{1000} \right) }^{1.0645} * 0.8586 = 20.0086
%
\end{equation}

So, according to COCOMO model and our estimation, almost exactly 20 person-months are needed to fully develop myTaxiService system.








\subsection{Schedule estimation}

Now, thanks to the previous results, it is possible to evaluate the time to develop ($ \mathbf{t} $). Time to develop is the calendar time in months that goes from the determination of the product's requirements to the completion of an acceptance activity certifying that the product satisfies its requirements.

The time to develop is given by this formula:\begin{equation}%
%
\mathbf{t} = 3.67 * {\left(\mathbf{e}\right)}^F \label{eqn:ttdev}
%
\end{equation}


Exponent $ F $ is defined as follows:
\begin{equation}
	F = 0.28 + 0.2 * \left( E - 0.91 \right) \label{eqn:expF}
\end{equation}  

By substituting the results from previous sections in \cref{eqn:ttdev,eqn:expF}, we obtain:\begin{gather}
%
	F = 0.28 + 0.2 * \left( 1.0645 - 0.91 \right) = 0.3109
	\\
	\mathbf{t} = 3.67 * {\left( 20.0086 \right)}^{0.3109} = 9.3157 \text{ m.} \simeq 40 \text{ w.} \label{eqn:timeVAL}
%
\end{gather}


The development of myTaxiService as a whole is expected to take more than 9 months.

By dividing the effort $ \mathbf{e} $ by the time to develop $ \mathbf{t} $, we get the estimated number of people needed for the project:\begin{equation}
%
	\mathbf{n} = \frac{\mathbf{e}}{\mathbf{t}} = \frac{20.0086}{9.3157} = 2.1478
%
\end{equation}

Some considerations about these figures will be done in the next chapter. 






