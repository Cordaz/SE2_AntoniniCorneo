\chapter{Cost and effort estimation}\label{chap:estimation}


\newcommand{\mSize}{\mathbf{s}}
\newcommand{\mFP}{\text{UFP}}
\newcommand{\mEffort}{\mathbf{e}}
\newcommand{\eaf}{\text{EAF}}  % Effort Adjustment Factor


\section{Function points}


\begin{equation}
\mFP = \sum \text{\#functions\_by\_type} * \text{weight}
\end{equation}



\newcommand{\myW}{1.5cm}
% TODO Format!
\begin{table}\begin{tabularx}{\textwidth}{ X C{\myW} C{\myW} C{\myW} }

\toprule
	
	\multirow{2}{*}{Function type} & \multicolumn{3}{c}{Complexity weight}\\

\cmidrule{2-4}%
	
	 & Simple & Average & Complex \\
		

\toprule

	External Input (EI)				& 3 	& 4 & 6 \\
\midrule
	External Output (EO)				& 4 & 5 & 7 \\
\midrule
	Internal Logical File (ILF)		& 7 & 10 & 15 \\
\midrule
	External Interface Files (EIF)	& 5 & 7 & 10 \\
\midrule
	External Inquiry (EQ)			& 3 	& 4 & 6 \\

\bottomrule


\end{tabularx}\end{table}




% http://csse.usc.edu/csse/research/COCOMOII/cocomo2000.0/CII_modelman2000.0.pdf
% http://www.qsm.com/resources/function-point-languages-table




\paragraph{External Input (EI)}

Count each unique user data or user control input type that enters the external boundary of the software system being measured.


\paragraph{External Output (EO)}

Count each unique user data or control output type that leaves the external boundary of the software system being measured.


\paragraph{Internal Logical File (ILF)}

Count each major logical group of user data or control information in the software system as a logical internal file type. Include each logical file (e.g., each logical group of data) that is generated, used, or maintained by the software system.


\paragraph{External Interface Files (EIF)} 

Files passed or shared between software systems should be counted as external interface file types within each system.


\paragraph{External Inquiry (EQ)}
 
Count each unique input-output combination, where input causes and generates an immediate output, as an external inquiry type.



\subsection{CALCOLO FINALE}

\begin{equation}
\mFP = \sum_{T} x_t
\end{equation}

\begin{quotation}
  It is like correcting the tallness of a person to relate it to a measure of his/her intelligence [Cit. Fenton]
\end{quotation}























\section{COCOMO}


In the following we are going to adopt the formulas defined in \mbox{\emph{COCOMO II.2000}} standard. Only a brief introduction of the theoretic meaning of the formulas is given. 

We are not going to present the whole tables from which values are extracted. Refer to the mentioned document for an insight on the COCOMO model. 


\subsection{Sizing the project}

A good size estimate is very important for a good model estimation. However, at this stage of myTaxiService project, this is a challenging operation, since it is nearly impossible to correctly guess in which proportion it will be composed of new code and of reused or readapted code.

That is why we are going to provide the \mbox{worst-case}, highly improbable, estimation: we assume that the whole code will be written from scratch. 

The quantity $\mSize$ is the estimated number of source lines of code (SLOC):
\begin{equation}
%
	\mSize = p * \mFP
%
\end{equation}

The parameter $ p = 46 $ (\url{http://www.qsm.com/resources/function-point-languages-table}) converts the function points to SLOC, in the specific case of J2EE (Java Enterprise Edition).

In our case the sizing of the project results:
\begin{equation}
%
	\mSize = 46 * YYY = XXX  % TODO Solve
%
\end{equation}



\subsection{Effort estimation}

Effort is expressed in terms of \mbox{person-months} (PM). A {person-month} is the amount of time one person spends working on the software development project for one month. COCOMO defines the following formula to estimate it:
\begin{equation}
%
	\mEffort = 2.94 * { \left( \frac{\mSize}{1000} \right) }^{E} * \eaf \label{eqn:effort}
%
\end{equation}


Exponent $ E $ and parameter $ \eaf $ are defined as follows:
\begin{gather}
%
	%E = exponent derived from Scale Drivers
	E = 0.91 + 0.01 * \sum_{i}^{5} \text{SF}_i \label{eqn:Eexp}\\%
%
	% EAF = Effort Adjustment Factor derived from Cost Drivers
	\eaf     = \prod_{i=1}^{n} \text{EM}_i \label{eqn:eaf}
%
\end{gather}




In the following paragraphs we are going to present the factors from which exponent $ E $ and parameter $ \eaf $ are derived. For a justification of the choices we made, please refer to \cref{chap:justification}.









\paragraph{Scale factors} Exponent $ E $ is an aggregation of five scale factors that account for some possible overheads encountered for software projects. Each factor has a range of rating levels, from \emph{Very Low} to \emph{Extra high}, and each rating level has a weight. The specific value of the weight is called a scale factor ($ \text{SF} $). In the following table we are going to present the adopted values.



\begin{table}\begin{tabularx}{\textwidth}{ >{\ttfamily}c X c c }

\toprule
\normalfont\textsc{ID} & \normalfont\textsc{Scale factor} & \normalfont\textsc{Level} & $ \text{SF} $ \\
\toprule
PREC	& Precedentedness			& Low		& $ 4.96 $ \\ \midrule
FLEX	& Development flexibility	& High		& $ 2.03 $ \\ \midrule
RESL	& Risk resolution			& Nominal	& $ 4.24 $\\ \midrule
TEAM	& Team cohesion				& Very high	& $ 1.10 $\\ \midrule
PMAT	& Process maturity			& High		& $ 3.12 $\\ 

\bottomrule
	
\end{tabularx}
\end{table}



As a result of the previous choices and on the basis of \cref{eqn:Eexp}, the parameter $ E $ results:
\begin{equation}
	E = 0.91 + 0.01 * 15.45 = 1.0645
\end{equation}

















\paragraph{Cost drivers} The parameter $ \eaf $ is derived from the effort multipliers ($ \text{EM} $) of the Cost drivers. Cost drivers are used to capture characteristics of the product under development, of the personnel working on it, and of general practices that affect the effort to complete the project.



\begin{table}\begin{tabularx}{\textwidth}{ >{\ttfamily}c X c c }

\toprule
\normalfont\textsc{ID} & \normalfont\textsc{Cost factor} & \normalfont\textsc{Level} & $ \text{EM} $ \\
\toprule

RELY	& Required software reliability	& Low		& $ 0.92 $ \\ \midrule
DATA	& Data base size					& Nominal	& $ 1.00 $ \\ \midrule
CPLX	& Product complexity				& High		& $ 1.17 $ \\ \midrule
RUSE	& Developed for reusability		& High		& $ 1.07 $ \\ \midrule
DOCU	& Documentation match to lifecycle needs		& High		& $ 1.11 $ \\ \midrule

TIME	& Execution time constraint		& --			& $  --  $ \\ \midrule
STOR	& Main storage constraint		& --			& $  --  $ \\ \midrule
PVOL	& Platform volatility			& Low		& $ 0.87 $ \\ \midrule

ACAP	& Analyst capability				& High		& $ 0.85 $ \\ \midrule
PCAP	& Programmer capability			& High		& $ 0.88 $ \\ \midrule
PCON	& Personnel continuity			& Very low	& $ 1.29 $ \\ \midrule
APEX	& Applications experience 		& Nominal	& $ 1.00 $ \\ \midrule
PLEX	& Platform Experience			& Nominal	& $ 1.00 $ \\ \midrule
LTEX	& Language and tool experience	& Nominal	& $ 1.00 $ \\ \midrule

TOOL	& Use of software tools			& Nominal	& $ 1.00 $ \\ \midrule
SITE	& Multisite development			& Extra high	& $ 0.80 $ \\ \midrule
SCED	& Required development schedule	& High		& $ 1.00 $ \\



\bottomrule
	
\end{tabularx}\end{table}







As a result of the previous choices, the parameter $ \eaf $, defined in \cref{eqn:eaf}, results:

\begin{equation}
	\eaf = 0.85858	
\end{equation}





\paragraph{Effort estimation}

Person/month.

Total:

\begin{equation}
%
	\mEffort = 2.94 * { \left( \frac{\mSize}{1000} \right) }^{E} * \eaf
%
\end{equation}









\subsection{Schedule estimation}

Duration:\begin{equation}%
%
\mathbf{d} = 3.67 * {\left(\mathbf{e}\right)}^F
%
\end{equation}

$ F = 0.28 + 0.2 * \left( E - B \right) $


Number of people:

\begin{equation}
\mathbf{n} = \frac{\mathbf{e}}{\mathbf{d}}
\end{equation}


%\subsection{The SCED (Schedule Constraints) Cost Driver}









