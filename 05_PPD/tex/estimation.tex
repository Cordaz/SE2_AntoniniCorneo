\chapter{Cost and effort estimation}\label{chap:estimation}


\newcommand{\mSize}{\mathbf{s}}
\newcommand{\mFP}{\text{FP}}
\newcommand{\mEffort}{\mathbf{e}}
\newcommand{\eaf}{\text{EAF}}  % Effort Adjustment Factor


\section{Function points}

The function point cost estimation approach is based on the amount of functionality in a software project and a set of individual project factors. In particular they measure a software project by quantifying the information processing functionality associated with major external data or control input, output, or file types. Five user function types should be identified, as defined in the following sections. 

At the end of quantification, the total number of function points ($ \mFP $) is given by the following formula:
\begin{equation}
\mFP = \sum \text{\#functions\_by\_type} * \text{weight} \label{eqn:fp}
\end{equation}


In \cref{tab:FPweights} the weights for each function type are presented, as a reference. Then, in the following sections, each type is analysed and the corresponding function points are calculated.

\newcommand{\myW}{1.5cm}

% TODO Togliere? Format!
\begin{table}\begin{tabularx}{\textwidth}{ >{\ttfamily}C{.6cm} X C{\myW} C{\myW} C{\myW} }

\toprule
	\multirow{2}{*}{\normalfont\textsc{ID}} &
	\multirow{2}{*}{\normalfont\textsc{Function type}} & \multicolumn{3}{c}{\normalfont\textsc{Complexity weight}}\\

\cmidrule{3-5}%
	
	 && Simple & Average & Complex \\
		

\toprule

	EI & External input				& $3$ 	& $4$	& $6$ \\
\midrule
	EO & External output				& $4$	& $5$	& $7$ \\
\midrule
	ILF & Internal logical file		& $7$	& $10$	& $15$ \\
\midrule
	EIF & External interface files	& $5$	& $7$	& $10$ \\
\midrule
	EQ & External inquiry			& $3$ 	& $4$	& $6$ \\

\bottomrule


\end{tabularx}

\caption{Function points complexity weights table.}
\label{tab:FPweights}

\end{table}





%TODO Total XX in \texttt!!!






\subsection*{External input}

%Count each unique user data or user control input type that enters the external boundary of the software system being measured.

Users interact with the application in many ways:
\begin{description}

	\item [Login, logout and registration] these are simple operations managed by simple components.
	
	\item [Memorise an address] simple operation.
	
	\item [Make a request, reservation] for each request or reservation the systems must interact with many entities. Moreover a taxi has to be allocated. These are complex operations, since many entities are involved.
	
	\item [Cancel a reservation] a simple operation that requires only an object elimination. 
	
	\item [Report taxi availability] this operation requires looking for the pertinence area of the taxi, in order to correctly enqueue it; this can be estimated as an average operation.
	
	\item [Taxi driver accepts or refuses a ride] the first case involves two simple operations, since the ride shall be initiated and a notification must be prepared. Instead, the second case is more complicated: whenever a ride is refused, the system makes a new allocation.

\end{description}

\newcommand{\myWFP}{2.5cm}

\begin{table*}\begin{tabularx}{\textwidth}{ X >{\itshape}C{\myWFP} C{\myWFP} C{\myWFP} }

\toprule
	
	\normalfont\textsc{Function} &
	\normalfont\textsc{Complexity}	& 
	\normalfont\textsc{Number} &
	\normalfont\textsc{Total weight} \\

\toprule

	Login, logout, registration	& Simple		& $ 3 $		& $ 3*3 = 9 $ \\
\midrule
	Address memorisation			& Simple		& $ 1 $		& $ 1*3 = 3 $ \\
\midrule
	Request, reservation			& Complex	& $ 3 $		& $ 3*6 = 18 $ \\
\midrule
	Cancel reservation			& Simple		& $ 1 $ 		& $ 1*3 = 3 $ \\
\midrule
	Report availability			& Average	& $ 1 $ 		& $ 1*4 = 4 $ \\
\midrule
	Ride accepted				& Simple		& $ 2 $ 		& $ 2*3 = 6 $ \\
\midrule
	Ride refused				& Complex	& $ 1 $ 		& $ 1*6 = 6 $ \\

\bottomrule

\normalfont\textsc{Total} EI && & $ 49 $ \\

\bottomrule


\end{tabularx}\end{table*}










\subsection*{External output}

%Count each unique user data or control output type that leaves the external boundary of the software system being measured.

The system shall provide notification to all types of users, which can be of various type (SMS, push, \dots). Each notification contains a small amount of data, however those must follow a precise protocol to be sent, which makes the operation a bit more complex: an average cost is to be considered.



\begin{table*}\begin{tabularx}{\textwidth}{ X >{\itshape}C{\myWFP} C{\myWFP} C{\myWFP} }

\toprule
	
	\normalfont\textsc{Function} &
	\normalfont\textsc{Complexity}	& 
	\normalfont\textsc{Number} &
	\normalfont\textsc{Total weight} \\

\toprule

	Notification	& Average		& $ 1 $		& $ 1*5 = 5 $ \\

\bottomrule

\normalfont\textsc{Total} EO && & $ 5 $ \\
\bottomrule


\end{tabularx}\end{table*}



\subsection*{Internal logical file}

%Count each major logical group of user data or control information in the software system as a logical internal file type. Include each logical file (e.g., each logical group of data) that is generated, used, or maintained by the software system.


The system includes many ILFs, as they are needed to store all the information about customers, areas and addresses, cabs and drivers, requests, reservations and allocations. In detail:

\begin{itemize}
	\item few information about customers are stored, in simple structures: first and last name, login credential and phone number. However, since registered customers' addresses are stored as well, it is appropriate to consider an average cost.
	\item to map the whole city in addresses and in their pertinence areas few data are needed: string and GPS data types may be sufficient to the purpose. However, every area contains a great amount of addresses, so an high complexity is to be accounted for.
	\item although many information about taxi drivers and taxi cabs are stored in the system, data needed for the operation of the system are more limited and very simple. Given that, we assign a low complexity level.
	\item the system needs to store many information for each request, reservation and many other information are needed to allocate the taxi. This requires a high complexity operation.
\end{itemize}






\begin{table*}\begin{tabularx}{\textwidth}{ X >{\itshape}C{\myWFP} C{\myWFP} C{\myWFP} }

\toprule
	
	\normalfont\textsc{Function} &
	\normalfont\textsc{Complexity}	& 
	\normalfont\textsc{Number} &
	\normalfont\textsc{Total weight} \\

\toprule

	Customers	& Average	& $ 1 $		& $ 1*10 = 10 $ \\
\midrule
	Areas and addresses		& Complex	& $ 2 $		& $ 2*15 = 30 $ \\
\midrule
	Taxis, taxi drivers		& Simple		& $ 1 $		& $ 1*7 = 7 $ \\
\midrule
	Requests, reservations, allocations	& Complex	& $ 3 $ 	& $ 3*15 = 45 $ \\

\bottomrule

\normalfont\textsc{Total} ILF && & $ 92 $ \\

\bottomrule


\end{tabularx}\end{table*}





\subsection*{External interface files} 

%Files passed or shared between software systems should be counted as external interface file types within each system.

However, myTaxiService system does not require external data, so this is to be set to $ 0 $.







\subsection*{External inquiry}
 
%Count each unique \mbox{input-output} combination, where input causes and generates an immediate output, as an external inquiry type.

Registered customers can access their profile and their memorised addresses. Both operations can be considered as simple.


\begin{table*}\begin{tabularx}{\textwidth}{ X >{\itshape}C{\myWFP} C{\myWFP} C{\myWFP} }

\toprule
	
	\normalfont\textsc{Function} &
	\normalfont\textsc{Complexity}	& 
	\normalfont\textsc{Number} &
	\normalfont\textsc{Total weight} \\

\toprule

	Profile				& Simple		& $ 1 $		& $ 1*3 = 3 $ \\
\midrule
	Memorise address	& Simple		& $ 1 $		& $ 1*3 = 3 $ \\


\bottomrule

\normalfont\textsc{Total} EQ && & $ 6 $ \\



\bottomrule


\end{tabularx}\end{table*}





\subsection{Total function points}

The results of the previous sections are summarised in \cref{tab:FPtotal}.

\begin{table}\begin{tabularx}{\textwidth}{ >{\ttfamily}C{.6cm} X C{\myW} }

\toprule
\normalfont\textsc{ID} & \normalfont\textsc{Function type} & $ \mFP $\\

\toprule

	EI & External input				& $49$ \\
\midrule
	EO & External output				& $5$ \\
\midrule
	ILF & Internal logical file		& $92$ \\
\midrule
	EIF & External interface files	& $0$ \\
\midrule
	EQ & External inquiry			& $6$\\

\bottomrule

\multicolumn{2}{l}{\normalfont\textsc{Total}} & $ 152 $ \\

\bottomrule

\end{tabularx}


\caption{Total function points.}
\label{tab:FPtotal}

\end{table}



By applying \cref{eqn:fp}, we get:
\begin{equation}
	\mFP = 107 % TODO correggi!
\end{equation}

Those we obtained are the Unadjusted Function Points (UFP). An attempt to adapt this result, which focuses on functionality, to the prediction of development costs, was made, but as it was wittily noted, ``it is like correcting the tallness of a person to relate it to a measure of his/her intelligence'' (N. Fenton), so we are not going to follow this road.























\section{COCOMO}

% TODO Definizione COCOMO!


In the following we are going to adopt the formulas defined in \mbox{\emph{COCOMO II.2000}} standard. Only a brief introduction of the theoretic meaning of the formulas is given. 

We are not going to present the whole tables from which values are extracted. Refer to the mentioned document for an insight on the COCOMO model. 


\subsection{Sizing the project}

A good size estimate is very important for a good model estimation. However, at this stage of myTaxiService project, this is a challenging operation, since it is nearly impossible to correctly guess in which proportion it will be composed of new code and of reused or readapted code.

That is why we are going to provide the \mbox{worst-case}, highly improbable, estimation: we assume that the whole code will be written from scratch. 

The quantity $\mSize$ is the estimated number of source lines of code (SLOC):
\begin{equation}
%
	\mSize = p * \mFP
%
\end{equation}

The parameter $ p = 46 $ (\url{http://www.qsm.com/resources/function-point-languages-table}) converts the function points to SLOC, in the specific case of J2EE (Java Enterprise Edition).

In our case the sizing of the project results:
\begin{equation}
%
	\mSize = 46 * YYY = XXX  % TODO Solve
%
\end{equation}



\subsection{Effort estimation}

Effort is expressed in terms of \mbox{person-months} (PM). A {person-month} is the amount of time one person spends working on the software development project for one month. COCOMO defines the following formula to estimate it:
\begin{equation}
%
	\mEffort = 2.94 * { \left( \frac{\mSize}{1000} \right) }^{E} * \eaf \label{eqn:effort}
%
\end{equation}


Exponent $ E $ and parameter $ \eaf $ are defined as follows:
\begin{gather}
%
	%E = exponent derived from Scale Drivers
	E = 0.91 + 0.01 * \sum_{i}^{5} \text{SF}_i \label{eqn:Eexp}\\%
%
	% EAF = Effort Adjustment Factor derived from Cost Drivers
	\eaf     = \prod_{i=1}^{n} \text{EM}_i \label{eqn:eaf}
%
\end{gather}




In the following paragraphs we are going to present the factors from which exponent $ E $ and parameter $ \eaf $ are derived. For a justification of the choices we made, please refer to \cref{chap:justification}.









\subsection*{Scale factors} Exponent $ E $ is an aggregation of five scale factors that account for some possible overheads encountered for software projects. Each factor has a range of rating levels, from \emph{Very Low} to \emph{Extra high}, and each rating level has a weight. The specific value of the weight is called a scale factor ($ \text{SF} $). In the following table we are going to present the adopted values.



\begin{table}\begin{tabularx}{\textwidth}{ >{\ttfamily}c X >{\itshape}c c }

\toprule
\normalfont\textsc{ID} & \normalfont\textsc{Scale factor} & \normalfont\textsc{Level} & $ \text{SF} $ \\
\toprule
PREC	& Precedentedness			& Low		& $ 4.96 $ \\ \midrule
FLEX	& Development flexibility	& High		& $ 2.03 $ \\ \midrule
RESL	& Risk resolution			& Nominal	& $ 4.24 $\\ \midrule
TEAM	& Team cohesion				& Very high	& $ 1.10 $\\ \midrule
PMAT	& Process maturity			& High		& $ 3.12 $\\ 

\bottomrule
	
\end{tabularx}
\end{table}



As a result of the previous choices and on the basis of \cref{eqn:Eexp}, the parameter $ E $ results:
\begin{equation}
	E = 0.91 + 0.01 * 15.45 = 1.0645
\end{equation}

















\subsection*{Cost drivers} The parameter $ \eaf $ is derived from the effort multipliers ($ \text{EM} $) of the Cost drivers. Cost drivers are used to capture characteristics of the product under development, of the personnel working on it, and of general practices that affect the effort to complete the project.



\begin{table}\begin{tabularx}{\textwidth}{ >{\ttfamily}c X >{\itshape}c c }

\toprule
\normalfont\textsc{ID} & \normalfont\textsc{Cost factor} & \normalfont\textsc{Level} & $ \text{EM} $ \\
\toprule

RELY	& Required software reliability	& Low		& $ 0.92 $ \\ \midrule
DATA	& Data base size					& Nominal	& $ 1.00 $ \\ \midrule
CPLX	& Product complexity				& High		& $ 1.17 $ \\ \midrule
RUSE	& Developed for reusability		& High		& $ 1.07 $ \\ \midrule
DOCU	& Documentation for lifecycle needs		& High		& $ 1.11 $ \\ \midrule

TIME	& Execution time constraint		& --			&    --    \\ \midrule
STOR	& Main storage constraint		& --			&    --    \\ \midrule
PVOL	& Platform volatility			& Low		& $ 0.87 $ \\ \midrule

ACAP	& Analyst capability				& High		& $ 0.85 $ \\ \midrule
PCAP	& Programmer capability			& High		& $ 0.88 $ \\ \midrule
PCON	& Personnel continuity			& Very low	& $ 1.29 $ \\ \midrule
APEX	& Applications experience 		& Nominal	& $ 1.00 $ \\ \midrule
PLEX	& Platform Experience			& Nominal	& $ 1.00 $ \\ \midrule
LTEX	& Language and tool experience	& Nominal	& $ 1.00 $ \\ \midrule

TOOL	& Use of software tools			& Nominal	& $ 1.00 $ \\ \midrule
SITE	& Multisite development			& Extra high	& $ 0.80 $ \\ \midrule
SCED	& Required development schedule	& High		& $ 1.00 $ \\



\bottomrule
	
\end{tabularx}\end{table}







As a result of the previous choices, the parameter $ \eaf $, defined in \cref{eqn:eaf}, results:

\begin{equation}
	\eaf = 0.85858	
\end{equation}





\subsection*{Effort estimation}

Person/month.

Total:

\begin{equation}
%
	\mEffort = 2.94 * { \left( \frac{\mSize}{1000} \right) }^{E} * \eaf
%
\end{equation}









\subsection{Schedule estimation}

Duration:\begin{equation}%
%
\mathbf{d} = 3.67 * {\left(\mathbf{e}\right)}^F
%
\end{equation}

$ F = 0.28 + 0.2 * \left( E - B \right) $


Number of people:

\begin{equation}
\mathbf{n} = \frac{\mathbf{e}}{\mathbf{d}}
\end{equation}


%\subsection{The SCED (Schedule Constraints) Cost Driver}









