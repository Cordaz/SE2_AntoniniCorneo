\chapter{Introduction}


\section{Purpose}
Improving the public transport is crucial for a modernizing city and Milan, as such, has already taken some steps in this direction. However, until now the taxi service has been overlooked, if not neglected. Recently, though, the government of the city requested a comprehensive study about possible ways to improve its taxi service. This document will outline a computer system which will hopefully help the city council to achieve this goal.

As a result, the natural addressee of this document is the municipal government itself, which will evaluate the feasibility of the project we are proposing. Once the project is approved, this document, along with other ones that will follow, will constitute a solid, non-technical guide to developers' work.


\section{Scope}
To be more specific, the council wants the access to the service to be simplified, as well as to guarantee a fair management of taxi queues, which will benefit both passengers and taxi drivers. 

These targets can be reached by implementing myTaxiService, an inclusive computer system which lets users make an immediate or delayed reservation for taxi rides, and manages the requests in order to minimise the waiting time for all passengers. Users will be able to access the new service through both a mobile application and a website; nevertheless, the current, phone-based system continues to be operative, in order to meet the needs of those people who are unfamiliar with the new service, and also of those who are not able to access it. 

Moreover, taxi drivers will be supported in their activity with a mobile application as well, which will run on their smartphone, letting them receive the requests for taxi rides.


\section{Definitions, acronyms, and abbreviations}
In order to avoid ambiguity, we find it appropriate to explicitly define some terms which will recur throughout the document:

\begin{description}
	\item [Customer] the person who makes use of the reservation service through one of the available methods (mobile application, website, phone call); customers can register on the service, thus gaining access to some convenient additional functions; otherwise they are considered as guests; they may be referred to as users as well; typically, but not necessarily, the person who made the reservation is also the person who actually benefits from the ride.
	\item [Mobile application] a program designed to work on mobile devices, namely smartphones and tablets; in the specific context of this document, we have two or them, since both the customers and the taxi drivers need one.
	\item [Website] in the specific context of this document, the website is a particular web page on the Internet, reachable with a browser (e.g. Internet Explorer, Google Chrome, Mozilla Firefox, \dots), by means of which a user can access the service.
	\item [System] in the specific context of this document, the system is the actual, computer-based provider of the service; notice that the two mobile applications, the website and phone calls are ways to access it.  
	\item [Taxi queue] list of taxis assigned to a specific area of the city; the policy that regulates the queue is a ``First In, First Out'' one, which means that the first taxi that is enlisted in the queue, is the first who receives a reservation, as well; there may be some exceptions, which will be described farther in this document.
	\item [Area] in order for the system to guarantee a fair management of the taxi queues, the city is divided in portions of approximately \SI{2}{\square\kilo\meter}, whose borders are well-defined by road intersections.
	\item [Request / Reservation] through the system, the customer can reserve a taxi ride; actually, his reservation may be either immediate (and we will call it request), meaning that a taxi cab is immediately allocated and will reach him as soon as possible, or delayed, which is to say that he books a ride by specifying its origin, destination and time.
\end{description}


\section{References}\label{sec:references}	
Farther in this writing we will refer to some external documents, which are listed below:
\begin{itemize}
	\item \emph{``Assignments 1 and 2''}, section~2, parts~I and~II; \printdate{2015-10-13}. This is to be regarded as the high level description of the problem. Available at: \url{https://beep.metid.polimi.it/documents/3343933/d5865f65-6d37-484e-b0fa-04fcfe42216d}.
	\item \emph{``Legge quadro per il trasporto di persone mediante autoservizi pubblici non di linea''}, law number 21/1992; \printdate{1992-01-15}. This is the current law that regulates the taxi service in Italy. Available (only in Italian) at: \url{http://www.normattiva.it/uri-res/N2Ls?urn:nir:stato:legge:1992-01-15;21}.
	\item \emph{``Codice in materia di protezione dei dati personali''}, law number 196/2003; \printdate{2003-06-30}. This is the current Italian law that regulates privacy issues. Available (only in Italian) at: \url{http://www.normattiva.it/uri-res/N2Ls?urn:nir:stato:decreto.legislativo:2003-06-30;196}.
\end{itemize}


\section{Overview of the document}	
This document develops as follows. In \cref{chap:overallDesc} we give a general description of the factors that affect both myTaxiService system and its requirements. In particular, this section focuses on the functions which are provided by the system, the constraints we have to impose, the assumptions we make. Requirements are thoroughly analysed in \cref{chap:requirements}, which is the most technical one. Formal and informal graphical representations are given as a support. Eventually, \cref{chap:appendix} contains some supporting information and material, including tables of contents and an index.


