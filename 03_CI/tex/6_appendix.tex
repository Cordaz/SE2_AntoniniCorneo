\chapter{Appendix}\label{chap:appendix}

\section{Hours of work}
The writing of this document took the following amount of time:

\begin{description}
	\item [Paolo Antonini] 38 hours.
	\item [Andrea Corneo] 30 hours.
\end{description}



\chapter{Reference - Java checklist}
\section{Naming Conventions}
1. All class names, interface names, method names, class variables, method variables, and constants used should have meaningful names and do what the name suggests.
2. If one-character variables are used, they are used only for temporary “throwaway” variables, such as those used in for loops.
3. Class names are nouns, in mixed case, with the first letter of each word in capitalized. Examples: class Raster; class ImageSprite;
4. Interface names should be capitalized like classes.
5. Method names should be verbs, with the first letter of each addition word
capitalized. Examples: getBackground(); computeTemperature().
6. Class variables, also called attributes, are mixed case, but might begin with an underscore (‘\_’) followed by a lowercase first letter. All the remaining words in the variable name have their first letter capitalized.
Examples: \_windowHeight, timeSeriesData.
7. Constants are declared using all uppercase with words separated by an
underscore. Examples: MIN\_WIDTH; MAX\_HEIGHT;
\section{Indention}
8. Three or four spaces are used for indentation and done so consistently
9. No tabs are used to indent
\section{Braces}
10. Consistent bracing style is used, either the preferred “Allman” style (first brace goes underneath the opening block) or the “Kernighan and Ritchie” style (first brace is on the same line of the instruction that opens the new block).
11.All if, while, do-while, try-catch, and for statements that have only one statement to execute are surrounded by curly braces. Example:
Avoid this:
if ( condition )
doThis(); Instead do this:
if ( condition ) {
doThis(); }
\section{File Organization}
12.Blank lines and optional comments are used to separate sections (beginning comments, package/import statements, class/interface declarations which include class variable/attributes declarations, constructors, and methods).
13. Where practical, line length does not exceed 80 characters.
14.When line length must exceed 80 characters, it does NOT exceed 120 characters.
Wrapping Lines
15. Line break occurs after a comma or an operator.
16. Higher-level breaks are used.
17. A new statement is aligned with the beginning of the expression at the
same level as the previous line.
Comments
18.Comments are used to adequately explain what the class, interface, methods, and blocks of code are doing.
19. Commented out code contains a reason for being commented out and a date it can be removed from the source file if determined it is no longer needed.
Java Source Files
20. Each Java source file contains a single public class or interface.
21. The public class is the first class or interface in the file.
22. Check that the external program interfaces are implemented consistently
with what is described in the javadoc.
23. Check that the javadoc is complete (i.e., it covers all classes and files part
of the set of classes assigned to you).
Package and Import Statements
24.If any package statements are needed, they should be the first non- comment statements. Import statements follow.
Class and Interface Declarations
25. The class or interface declarations shall be in the following order:
A. class/interface documentation comment
B. class or interface statement
C. class/interface implementation comment, if necessary
D. class (static) variables
a. first public class variables
b. next protected class variables
c. next package level (no access modifier)
d. last private class variables
E. instance variables
a. first public instance variables
e. next protected instance variables
f. next package level (no access modifier)
g. last private instance variables
F. constructors
G. methods
26.Methods are grouped by functionality rather than by scope or
accessibility.
27.Check that the code is free of duplicates, long methods, big classes,
breaking encapsulation, as well as if coupling and cohesion are adequate.
Initialization and Declarations
28. Check that variables and class members are of the correct type. Check that they have the right visibility (public/private/protected)
29. Check that variables are declared in the proper scope
30. Check that constructors are called when a new object is desired
31. Check that all object references are initialized before use
32. Variables are initialized where they are declared, unless dependent upon
a computation
33.Declarations appear at the beginning of blocks (A block is any code
surrounded by curly braces “{“ and “}” ). The exception is a variable can be declared in a ‘for’ loop.
Method Calls
34. Check that parameters are presented in the correct order
35. Check that the correct method is being called, or should it be a different
method with a similar name
36. Check that method returned values are used properly
Arrays
37. Check that there are no off-by-one errors in array indexing (that is, all required array elements are correctly accessed through the index)
38.Check that all array (or other collection) indexes have been prevented from going out-of-bounds
39. Check that constructors are called when a new array item is desired
Object Comparison
40. Check that all objects (including Strings) are compared with "equals" and not with "=="
Output Format
41. Check that displayed output is free of spelling and grammatical errors
42. Check that error messages are comprehensive and provide guidance as to
how to correct the problem
43. Check that the output is formatted correctly in terms of line stepping and
spacing
Computation, Comparisons and Assignments
44.Check that the implementation avoids “brutish programming: (see http://users.csc.calpoly.edu/~jdalbey/SWE/CodeSmells/bonehead.html )
 
45.Check order of computation/evaluation, operator precedence and parenthesizing
46. Check the liberal use of parenthesis is used to avoid operator precedence problems.
47. Check that all denominators of a division are prevented from being zero
48. Check that integer arithmetic, especially division, are used appropriately
to avoid causing unexpected truncation/rounding
49. Check that the comparison and Boolean operators are correct
50.Check throw-catch expressions, and check that the error condition is
actually legitimate
51. Check that the code is free of any implicit type conversions
Exceptions
52. Check that the relevant exceptions are caught
53. Check that the appropriate action are taken for each catch block
Flow of Control
54.In a switch statement, check that all cases are addressed by break or return
55. Check that all switch statements have a default branch
56.Check that all loops are correctly formed, with the appropriate
initialization, increment and termination expressions
Files
57. Check that all files are properly declared and opened
58. Check that all files are closed properly, even in the case of an error
59. Check that EOF conditions are detected and handled correctly
60. Check that all file exceptions are caught and dealt with accordingly


\chapter{Assigned code}

%TODO Move to .java files.

\begin{lstlisting}[language=Java]
   /* Evaluates a client's conformance to a security policies
    * at the client authentication layer.
    *
    * returns true if conformant ; else returns false
    */
    private boolean evaluate_client_conformance_ascontext(
                        SecurityContext ctx,
                        EjbIORConfigurationDescriptor iordesc,
                        String realmName)
    {

        boolean client_authenticated = false;

        // get requirements and supports at the client authentication layer
        AS_ContextSec ascontext = null;
        try {
            ascontext = this.getCtc().createASContextSec(iordesc, realmName);
        } catch (Exception e) {
            _logger.log(Level.SEVERE, "iiop.createcontextsec_exception",e);

            return false;
        }
   

        /*************************************************************************
         * Conformance Matrix:
         *
         * |------------|---------------------|---------------------|------------|
         * | ClientAuth | targetrequires.ETIC | targetSupports.ETIC | Conformant |
         * |------------|---------------------|---------------------|------------|
         * |     Yes    |          0          |      1              |    Yes     |
         * |     Yes    |          0          |      0              |    No      |
         * |     Yes    |          1          |      X              |    Yes     |
         * |     No     |          0          |      X              |    Yes     |
         * |     No     |          1          |      X              |    No      |
         * |------------|---------------------|---------------------|------------|
         *
         * Abbreviations: ETIC - EstablishTrusInClient
         * 
         *************************************************************************/

        if  ( (ctx != null) && (ctx.authcls != null) &&  (ctx.subject != null))
            client_authenticated = true;
        else
            client_authenticated = false;

        if (client_authenticated) {
            if ( ! ( isSet(ascontext.target_requires, EstablishTrustInClient.value)
                     || isSet(ascontext.target_supports, EstablishTrustInClient.value))){
            return false; // non conforming client
            }
            // match the target_name from client with the target_name in policy 

            byte [] client_tgtname = getTargetName(ctx.subject);

            if (ascontext.target_name.length != client_tgtname.length){
                return false; // mechanism did not match.
            }            
            for (int i=0; i < ascontext.target_name.length ; i ++)
                if (ascontext.target_name[i] != client_tgtname[i]){
                    return false; // mechanism did not match
                }
        } else { 
            if ( isSet(ascontext.target_requires, EstablishTrustInClient.value)){
                return false;  // no mechanism match.
            }
        }
        return true;
    }	
\end{lstlisting}

\begin{lstlisting}
	   /* Evaluates a client's conformance to a security policy
    * at the sas context layer. The security policy
    * is derived from the EjbIORConfigurationDescriptor.
    *
    * returns true if conformant ; else returns false
    */
    private boolean evaluate_client_conformance_sascontext(
                        SecurityContext ctx,
                        EjbIORConfigurationDescriptor iordesc)
    {

        boolean caller_propagated = false;

        // get requirements and supports at the sas context layer
        SAS_ContextSec sascontext = null;
        try {
            sascontext = this.getCtc().createSASContextSec(iordesc);
        } catch (Exception e) {
            _logger.log(Level.SEVERE,"iiop.createcontextsec_exception",e);
            return false;
        }

            
        if  ( (ctx != null) && (ctx.identcls != null) &&  (ctx.subject != null))
            caller_propagated = true;
        else
            caller_propagated = false;

        if (caller_propagated) {
            if ( ! isSet(sascontext.target_supports, IdentityAssertion.value))
                return false; // target does not support IdentityAssertion

            /* There is no need further checking here since SecServerRequestInterceptor
             * code filters out the following:
             * a. IdentityAssertions of types other than those required by level 0 
             *    (for e.g. IdentityExtension)
             * b. unsupported identity types.
             * 
             * The checks are done in SecServerRequestInterceptor rather than here
             * to minimize code changes.
             */
            return true;
        }
        return true; //  either caller was not propagated or mechanism matched.
    }
\end{lstlisting}


\begin{lstlisting}
	    /**
     * Evaluates a client's conformance to the security policies configured
     * on the target.
     * Returns true if conformant to the security policies
     * otherwise return false.
     *
     * Conformance checking is done as follows:
     * First, the object_id is mapped to the set of EjbIORConfigurationDescriptor.
     * Each EjbIORConfigurationDescriptor corresponds to a single CompoundSecMechanism
     * of the CSIv2 spec. A client is considered to be conformant if a
CompoundSecMechanism
     * consistent with the client's actions is found i.e. transport_mech,
as_context_mech
     * and sas_context_mech must all be consistent.
     * 
     */
    private boolean evaluate_client_conformance(SecurityContext   ctx,
                                                byte[]            object_id,
                                                boolean           ssl_used,
                                                X509Certificate[] certchain)
    {
        // Obtain the IOR configuration descriptors for the Ejb using
        // the object_id within the SecurityContext field.

        // if object_id is null then nothing to evaluate. This is a sanity
        // check - for the object_id should never be null.
        
        if (object_id == null)
            return true;

        if (protocolMgr == null)
            protocolMgr = orbHelper.getProtocolManager();

        // Check to make sure protocolMgr is not null. 
        // This could happen during server initialization or if this call
        // is on a callback object in the client VM. 
        if (protocolMgr == null)
            return true;

        EjbDescriptor ejbDesc = protocolMgr.getEjbDescriptor(object_id);

        Set iorDescSet = null;
        if (ejbDesc != null) {
	    iorDescSet = ejbDesc.getIORConfigurationDescriptors();
	}
	else {
	    // Probably a non-EJB CORBA object.
	    // Create a temporary EjbIORConfigurationDescriptor.
	    iorDescSet = getCorbaIORDescSet();
	}

	if(_logger.isLoggable(Level.FINE)) {
	    _logger.log(Level.FINE,
			"SecurityMechanismSelector.evaluate_client_conformance: iorDescSet: " + iorDescSet);
	}

        /* if there are no IORConfigurationDescriptors configured, then
         * no security policy is configured. So consider the client 
         * to be conformant.
         */
        if (iorDescSet.isEmpty())
            return true;

        // go through each EjbIORConfigurationDescriptor trying to find
        // a find a CompoundSecMechanism that matches client's actions.
        boolean checkSkipped = false;
        for (Iterator itr = iorDescSet.iterator(); itr.hasNext();) {
            EjbIORConfigurationDescriptor iorDesc = 
                (EjbIORConfigurationDescriptor) itr.next();
            if(skip_client_conformance(iorDesc)){
		if(_logger.isLoggable(Level.FINE)) {
		    _logger.log(Level.FINE,
				"SecurityMechanismSelector.evaluate_client_conformance: skip_client_conformance");
		}
                checkSkipped = true;
                continue;
            }
            if (! evaluate_client_conformance_ssl(iorDesc, ssl_used, certchain)){
		if(_logger.isLoggable(Level.FINE)) {
		    _logger.log(Level.FINE,
				"SecurityMechanismSelector.evaluate_client_conformance: evaluate_client_conformance_ssl");
		}
                checkSkipped = false;
                continue;
            }
            String realmName = "default";
            if(ejbDesc != null && ejbDesc.getApplication() != null) {
                realmName = ejbDesc.getApplication().getRealm();
            }
            if(realmName == null) {
                realmName = iorDesc.getRealmName();
            }
            if (realmName == null) {
                realmName = "default";
            }
            if ( ! evaluate_client_conformance_ascontext(ctx, iorDesc ,realmName)){
		if(_logger.isLoggable(Level.FINE)) {
		    _logger.log(Level.FINE,
				"SecurityMechanismSelector.evaluate_client_conformance: evaluate_client_conformance_ascontext");
		}
                checkSkipped = false;
                continue;
            }
            if  ( ! evaluate_client_conformance_sascontext(ctx, iorDesc)){
		if(_logger.isLoggable(Level.FINE)) {
		    _logger.log(Level.FINE,
				"SecurityMechanismSelector.evaluate_client_conformance: evaluate_client_conformance_sascontext");
		}
               checkSkipped = false;
               continue;
            }
            return true;  // security policy matched.
        }
        if(checkSkipped)
            return true;
        return false; // No matching security policy found
    }
\end{lstlisting}




