\chapter{Problems}

%List of issues found by applying the checklist: <report the classes/code fragments that do not fulfill some points in the check list. Explain which point is not fulfilled and why>.

%Any other problem you have highlighted: <list here all the parts of code that you think create or may create a bug and explain why>.


\lipsum[1]


\section{\normalfont\texttt{evaluate\_client\_conformance\_ascontext}}

The objective of the method is described by the preceding comment. The code of the method has been split into sections, in order to improve readability. Following each section, a brief report of the violated rules (a complete reference is in \cref{chap:reference}).

\inputpartmint{startASC}{11215}

\begin{description}
	
	\item [\labelcref{item:5}]
		the name of the method \verb|evaluate_client_conformance_ascontext| (\lmref{l:11209}) does not comply with rule, that is it does not match the regular expression \verb|^[a-z][a-zA-Z0-9]*$|; since other methods in the class do follow the same pattern (lowercase words, separated by underscores), maybe this is done intentionally to improve readability; a compliant solution would be: \texttt{evaluateClientConformanceAscontext}. %TODO Check acapo. Maybe spiegare per forzare linea successiva?
	
	\item [\labelcref{item:6}]
		the variable name \verb|client_authenticated| (\lmref{l:11215}) does not comply with the rule; a compliant solution would be \verb|clientAuthenticated|. 
	
	\item [\labelcref{item:10}]
		since we choose to adopt K\&R style, since it is compact yet tidy, only \lmref{l:11213} does not comply (the opening brace should be placed at the end of the method declaration).
		
\end{description}

\inputpartmint{11217}{11225}

\begin{description}
	
	\item [\labelcref{item:52}]
		the \verb|try-catch| group in \lmrange{l:11219}{l:11225} is roughly managed: catching, generically, \verb|(Exception e)| does not allow to provide a detailed log of the error.

\end{description}


\inputpartmint{11228}{11248}

\begin{description}
	
	\item [\labelcref{item:11}] 
		both \verb|if| and \verb|else| clauses in \lmrange{l:11245}{l:11248} do not surround their statement with braces, which is to be avoided.
	
	\item [\labelcref{item:44}]
		to avoid an example of ``brutish programming'', \lmrange{l:11245}{l:11248}, together with \lmref{l:11215}, can be collapsed to the following statement: 
		\begin{minted}[showtabs=false, linenos=false,frame=leftline]{java}
		boolean client_authenticated = (ctx != null) && (ctx.authcls != null) && (ctx.subject != null);
		\end{minted}
		however this line is too long (it exceeds the 80 character limit stated in \cref{item:13}) and may be less readable; moreover, future changes to the method may take more time, should the truth condition be expanded, or the \verb|if| clause used again.
\end{description}

\inputpartmint{11250}{endASC}

\begin{description}
	
	\item [\labelcref{item:6}]
		the \verb|client_tgtname| variable, declared in \lmref{l:11257}, does not comply with standard naming rules; a compliant name for the variable would be: \verb|clientTgtname|.
	
	\item [\labelcref{item:11}] 
		the \verb|for| group in \lmrange{l:11262}{l:11265} does not surround its statement with braces, which is to be avoided.
	
	\item [\labelcref{item:13}]
		sometimes it is highly improbable to respect the 80 characters limit, when it comes to nested if clauses which require the execution of other methods; however, in this fragment only a very limited number of lines exceeds significantly that limit (e.g., \lmref{l:11252}).
	
	\item [\labelcref{item:15}]
		the condition of the \verb|if| statement in \lmrange{l:11251}{l:11252} breaks before \verb|||| operator, instead of after, which is preferable.
	
\end{description}




















%\clearpage%%%%%%%%%%%%%%%%%%%%%%%%%%%%%%%%%%%%%%%%%%%%%%%%%%%%%%%%%%%%%%%%%%
\section{\normalfont\texttt{evaluate\_client\_conformance\_sascontext}}

\inputpartmint{startSAS}{21285}

\begin{description}
	
	\item [\labelcref{item:5}]
		the name of the method \verb|evaluate_client_conformance_sascontext| (\lmref{l:21280}) does not comply with rule, that is it does not match the regular expression \verb|^[a-z][a-zA-Z0-9]*$|; since other methods in the class do follow the same pattern (lowercase words, separated by underscores), maybe this is done intentionally to improve readability; a compliant solution would be: \texttt{evaluateClientConformanceSascontext}. %TODO Check acapo. Maybe spiegare per forzare linea successiva?
	
	\item [\labelcref{item:6}]
		the variable name \verb|caller_propagated| (\lmref{l:21285}) does not comply with the rule; a compliant solution would be \verb|callerPropagated|. 
	
	\item [\labelcref{item:10}]
		since we choose to adopt K\&R style, since it is compact yet tidy, only \lmref{l:21283} does not comply (the opening brace should be placed at the end of the method declaration).
		
\end{description}


\inputpartmint{21287}{21294}


\begin{description}
	
	\item [\labelcref{item:52}]
		the \verb|try-catch| group in \lmrange{l:21289}{l:21294} is roughly managed: catching, generically, \verb|(Exception e)| does not allow to provide a detailed log of the error.

\end{description}


\inputpartmint{21297}{21300}

\begin{description}
	
	\item [\labelcref{item:11}] 
		both \verb|if| and \verb|else| clauses in \lmrange{l:11245}{l:11248} do not surround their statement with braces, which is to be avoided.
		
	\item [\labelcref{item:44}]
		to avoid an example of ``brutish programming'', \lmrange{l:21297}{l:21300}, together with \lmref{l:21285}, can be collapsed to the following statement: 
		\begin{minted}[showtabs=false, linenos=false,frame=leftline]{java}
		boolean caller_propagated = (ctx != null) && (ctx.identcls != null) && (ctx.subject != null);
		\end{minted}
		however this line is too long (it exceeds the 80 character limit stated in \cref{item:13}) and may be less readable; moreover, future changes to the method may take more time, should the truth condition be expanded, or the \verb|if| clause used again.

\end{description}

\inputpartmint{21302}{endSAS}

\begin{description}
		
	\item [\labelcref{item:11}] 
		the \verb|if| group in \lmrange{l:21303}{l:21304} does not surround its statement with braces, which is to be avoided. 

\end{description}

	












































%\clearpage%%%%%%%%%%%%%%%%%%%%%%%%%%%%%%%%%%%%%%%%%%%%%%%%%%%%%%%%%%%%%%%%%%
\section{\normalfont\texttt{evaluate\_client\_conformance}}
\inputpartmint{startECC}{31342}
\begin{description}
	
	\item [\labelcref{item:5}] 
		in \lmref{l:31338} maybe to highlight pattern? 
			OR: \texttt{evaluate\_client\_conformance}
			SS: \texttt{evaluateClientConformance}
	
	\item [\labelcref{item:6}] 
		in \lmref{l:31339} 
			OR: \texttt{object\_id}
			SS: \texttt{objectId}; 
		in \lmref{l:31340} 
			OR: \texttt{ssl\_used}
			SS: \texttt{sslUsed}
	
		
	\item [\labelcref{item:10}]
		adopting K\&R style, only \lmref{l:31342} does not comply
	
\end{description}


\inputpartmint{31343}{31376}

\begin{description}
	
	\item [\labelcref{item:11}] \lmrange{l:31349}{l:31350}, \lmrange{l:31352}{l:31353}, \lmrange{l:31358}{l:31359}

\end{description}

\inputpartmint{31378}{endECC}


\begin{description}
		
	\item [\labelcref{item:11}] \lmrange{l:31382}{l:31383}, \lmrange{l:31435}{l:31436}

\end{description}








\subsection{General}

\begin{description}
	\item [\labelcref{item:9}]
		throughout the method, highlighted by $-\hspace{-0.4em}\rangle\hspace{-0.2em}|$ symbol. %TODO Sostituire! 

\end{description}






