\chapter{Requirements traceability} \label{chap:traceability}
To conclude this document, we would like to highlight some significant correspondences between this document and the previous \hyperlink{subsec:references}{\emph{Requirement analysis and specification document}}. Indeed, functional and non-functional requirements expressed there were kept as a guide while writing this document.

\section{Functional requirements}
Functional requirements are presented in a compact way in the \rasd in section~3.2. 

The component view (\cref{sec:componentView,sec:componentInterfaces}) was designed bearing in mind what was stated there. Every component contributes to comply with those requirements (most of the correlations are trivial). Requirement number 6 was particularly useful to design the main algorithms of the system (\cref{chap:algorithm}).

\section{Non-functional requirements}
The architecture presented in \cref{sec:highlevel,sec:styles} wants to be the counterpart of sections~3.3 to~3.6 in the \rasd, where non-functional requirements were outlined. It was designed to comply as much as possible to what was stated in those sections.

Nevertheless, there are some non-functional requirements which cannot be defined by any architectural decision (for instance, maintainability of the code, defined in paragraph~3.6.4 in the \rasd).